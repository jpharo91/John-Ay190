\documentclass[11pt,letterpaper]{article}

% Load some basic packages that are useful to have
% and that should be part of any LaTeX installation.
%
% be able to include figures
\usepackage{graphicx}
% get nice colors
\usepackage{xcolor}

% change default font to Palatino (looks nicer!)
\usepackage[latin1]{inputenc}
\usepackage{mathpazo}
\usepackage[T1]{fontenc}
% load some useful math symbols/fonts
\usepackage{latexsym,amsfonts,amsmath,amssymb}

% comfort package to easily set margins
\usepackage[top=1in, bottom=1in, left=1in, right=1in]{geometry}

% control some spacings
%
% spacing after a paragraph
\setlength{\parskip}{.15cm}
% indentation at the top of a new paragraph
\setlength{\parindent}{0.0cm}


\begin{document}

\begin{center}
\Large
Ay190 -- Worksheet 13\\
John Pharo\\
Date: \today\\
Supernova Remnants: Matthias Raives, Mee Chatarin Wongurailertkun
\end{center}

\section*{Problem 1}

I have familiarized myself.

\section*{Problem 2}

I simulated the Sun-Earth system for 4 years. The radius of the orbit remained basically constant over the course of the simulation.

\section*{Problem 3}

With 10$^4$ points over 4 years, the magnitude of the largest change in energy is around 2.76*10$^{37}$, which amounts to about a tenth of a percent of the energy in the system. As the resolution increases, this magnitude decreases even further.

\section*{Problem 4}

Over the century, the system changes energy by about 7*10$^{50}$ ergs. This is a large fraction of the energy remaining in the system, which is much worse than in the previous problems, but that is to be expected, since the simplistic n-body model begins to fail in more complicated systems. \\

This can further be seen by comparing the trajectories in stars13.dat. The simulation diverges from observation rather quickly.

\end{document}
