\documentclass[11pt,letterpaper]{article}

% Load some basic packages that are useful to have
% and that should be part of any LaTeX installation.
%
% be able to include figures
\usepackage{graphicx}
% get nice colors
\usepackage{xcolor}

% change default font to Palatino (looks nicer!)
\usepackage[latin1]{inputenc}
\usepackage{mathpazo}
\usepackage[T1]{fontenc}
% load some useful math symbols/fonts
\usepackage{latexsym,amsfonts,amsmath,amssymb}

% comfort package to easily set margins
\usepackage[top=1in, bottom=1in, left=1in, right=1in]{geometry}

% control some spacings
%
% spacing after a paragraph
\setlength{\parskip}{.15cm}
% indentation at the top of a new paragraph
\setlength{\parindent}{0.0cm}


\begin{document}

\begin{center}
\Large
Ay190 -- Worksheet 11\\
John Pharo\\
Date: \today\\
Lords of Waterdeep: Mee Chatarin Wongurailertkun, David Vartanyan
\end{center}

\section*{Problem 2}

\begin{figure}[bth]
\centering
\includegraphics[width=0.5\textwidth]{Upwind.pdf}
\caption{This is the plot of the upwind model using $\sigma = \sqrt{15}$.}
\label{fig:simpleplot}
\end{figure}

\begin{figure}[bth]
\centering
\includegraphics[width=0.5\textwidth]{Upwind2.pdf}
\caption{This is the plot of the upwind model where $\sigma = \sqrt{15}/5$. Notice that, after 2000 iterations, the numerical solution diverges pretty strongly, especially about the center.}
\label{fig:simpleplot}
\end{figure}

\section*{Problem 3}

\begin{figure}[bth]
\centering
\includegraphics[width=0.5\textwidth]{FTCS4.pdf}
\caption{This is the plot of the FTCS model for $n=500$.}
\label{fig:simpleplot}
\end{figure}

\begin{figure}[bth]
\centering
\includegraphics[width=0.5\textwidth]{FTCS3.pdf}
\caption{This is the plot of the FTCS model for $n=1000$.}
\label{fig:simpleplot}
\end{figure}

\begin{figure}[bth]
\centering
\includegraphics[width=0.5\textwidth]{FTCS2.pdf}
\caption{This is the plot of the FTCS model for $n=1500$.}
\label{fig:simpleplot}
\end{figure}

\begin{figure}[bth]
\centering
\includegraphics[width=0.5\textwidth]{FTCS.pdf}
\caption{This is the plot of the FTCS model for $n=2000$.}
\label{fig:simpleplot}
\end{figure}

At first, the numerical solution stays relatively close to the analytical result, but at large numbers of iterations, the two diverge significantly, and the numerical solution becomes pretty unstable.

\section*{Problem 4}

\begin{figure}[bth]
\centering
\includegraphics[width=0.5\textwidth]{LF2.pdf}
\caption{This is the plot of the Lax-Friedrich model for $n=1500$.}
\label{fig:simpleplot}
\end{figure}

\begin{figure}[bth]
\centering
\includegraphics[width=0.5\textwidth]{LF3.pdf}
\caption{This is the plot of the Lax-Friedrich model for $n=1000$.}
\label{fig:simpleplot}
\end{figure}

\begin{figure}[bth]
\centering
\includegraphics[width=0.5\textwidth]{LF4.pdf}
\caption{This is the plot of the Lax-Friedrich model for $n=500$.}
\label{fig:simpleplot}
\end{figure}

The Lax-Friedrich model begins to diverge from the analytical result at about $n=500$, producing a shorter and wider Gaussian profile.

\end{document}
