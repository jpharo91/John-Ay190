\documentclass[11pt,letterpaper]{article}

% Load some basic packages that are useful to have
% and that should be part of any LaTeX installation.
%
% be able to include figures
\usepackage{graphicx}
% get nice colors
\usepackage{xcolor}

% change default font to Palatino (looks nicer!)
\usepackage[latin1]{inputenc}
\usepackage{mathpazo}
\usepackage[T1]{fontenc}
% load some useful math symbols/fonts
\usepackage{latexsym,amsfonts,amsmath,amssymb}

% comfort package to easily set margins
\usepackage[top=1in, bottom=1in, left=1in, right=1in]{geometry}

% control some spacings
%
% spacing after a paragraph
\setlength{\parskip}{.15cm}
% indentation at the top of a new paragraph
\setlength{\parindent}{0.0cm}


\begin{document}

\begin{center}
\Large
Newtonian Binary Coalescence
Ay190 -- Final Project\\
Cutter Coryell and John Pharo\\
Date: \today
\end{center}

\section{Introduction}

In the paper \textit{Gravitational Radiation and the Motion of Two Point Masses} (Peters 1964), the equations of general relativity are used to obtain equations of conservation of energy, momentum, and angular momentum. From this, Peters determines the energy loss and angular momentum loss of a binary system of compact objects due to radiating gravitational waves. This yields the equations

$$ \frac{dE}{dt} = \frac{G}{5c^5} \left( \frac{\partial^3 \bar{I}_{ij}}{\partial t^3} \right)^2 \hspace*{.5in} \frac{dL_z}{dt} = \frac{2G}{5c^5} \epsilon_{zjk} \left( \frac{\partial^3 \bar{I}_{ij}}{\partial t^3} \frac{\partial^3 \bar{I}_{km}}{\partial t^3} \right) $$

where $\bar{I}$ is the reduced quadrupole tensor (with separation vector $\vec{x} (t)$ and reduced mass $\mu$):

$$ \bar{I}_{ij}  =\mu (x_i x_j - \frac{1}{3} \delta_{ij} x_k x_k) $$

The purpose of this project is to simulate the orbit and gravitational wave production of such a binary system. This can be done through numerical evaluation of the eccentricity and semi-major axis of the system. Derived in Peters Equations 5.6 and 5.7:

$$ \left< \frac{da}{dt} \right> = - \frac{64}{5} \cdot \frac{G^3 m_1 m_2 (m_1 + m_2)}{c^5 a^3 (1 - e^2)^{7/2}} \left( 1 + \frac{73}{24} e^2 + \frac{37}{96} e^4 \right) $$

$$ \left< \frac{de}{dt} \right> = - \frac{304}{15} \cdot e \cdot \frac{G^3 m_1 m_2 (m_1 + m_2)}{c^5 a^4 (1 - e^2)^{5/2}} \left( 1 + \frac{121}{304}e^2 \right) $$

Furthermore, we may derive the true anomaly $\varphi$:

$$ \frac{d \varphi}{dt} = \frac{L}{r^2}, \hspace*{.3in} e = \sqrt{1 + \frac{2EL^2}{(GM)^2}} \implies \frac{d \varphi}{dt} = \frac{\sqrt{GMa(1 - e^2)}}{r^2} $$

Through the use of Runge-Kutta integration, we may numerically evaluate these three derivatives in order to track the orbit of the system over time. Furthermore, by updating the separation vector, one can track the change in the reduced quadrupole over time, which the gravitational wave strain:

$$ h_+ = \frac{1}{r} \left( \ddot{\bar{I}}_{xx} - \ddot{\bar{I}}_{yy} \right) \hspace*{.5in} h_{\times} = \frac{2}{r} \ddot{\bar{I}}_{xy} $$

\section{The Code}

This project uses two main sections of code, one for the actual simulation of the binary system and one for analysis and production of plots of the data.

\subsection{Simulation}

The simulation code \texttt{simulate.py} is responsible for finding the numerical solutions for the binary system and generating the simulated orbits and gravitational waves produced over a given time scale. \\

The first section of the code sets various constants and parameters for the system. This includes the initial data settings for simulating the Hulse-Taylor binary and several important parameters for running the simulation. Of particular importance here are the names of the input and output directories (depending on whether a simulation is being continued from a previous run), the ending time for the simulation, and the number of time points used in the simulation (these last two determined the fixed time-step size). \\

The next two sections are a series of helper functions and the main code body itself. The main code begins by either opening a previous simulation to continue where it left off, or by establishing a new set of initial conditions for a new simulation. This done, the code enters the main loop of the program, which will terminate upon reaching the end of the established number of points. \\

First, the loop calls a helper function to perform RK4 integration on $a$, $e$, and $\varphi$. The RK4 integrator itself uses several helper functions, each representing the right-hand side of the equations presented in the Introduction. Having calculated the values of $a$, $e$, and $\varphi$ at the current time, the code then calls helper functions to calculate the new separation vector and reduced quadrupole tensor, and from these it calculates the wave strain. \\

The code can be set to produce a 3-D display of the orbit as it is calculated; if so, it is at this point in the code that the plot updates. However, it should be noted that running the display slows the simulation down significantly. After this, the loop writes the newly generated data to the appropriate output directory. Every time ten output files have been written, the loop prints an update on its progress to \texttt{stdout}.

\subsection{Analysis}

The analysis code \texttt{analyze.py} is responsible for taking the output generated by the simulation and plotting it in a useful fashion. The first section sets general useful constants in CGS. The second section takes important parameters for analyzing the simulation data, especially the name of the directory containing the data to be analyzed and the number of iterations it contains. \\

The next two sections load the existing data into lists and then use it to calculate the z-component of the orbital angular momentum and the total mechanical energy of the system, using the following equations:

$$ L_z = \sqrt{GMa (1 - e^2)} \hspace*{.5in} E = \frac{1}{2} \left( \frac{GM}{L_z} \right)^2 (e^2 - 1) $$

Finally, the code generates plots of the semi-major axis, eccentricity, energy, and angular momentum versus time and saves the figures to a new output directory. In order to make pyplot plots intelligible, sometimes the data lists are converted to reduced units. This should be made clear in the captions of the plots.

\section{Test and Results}

\subsection{Test Run}

Upon completing the simulation code, we first ran a test simulation using the parameters of the PSR B1913+16 binary system, also known as the Hulse-Taylor binary. Figures 1 through 4 illustrate the details of the simulation. The simulation predicted an inspiral time of about 300 million years, which matches other calculations for the Hulse-Taylor system. This provides reasonable confirmation of the success of the algorithm used in our simulation. Next, we applied the code to more useful scenarios.

\begin{figure}[bth]
\centering
\includegraphics[width=0.5\textwidth]{billion_figs/semi-major_axis.pdf}
\caption{This is the plot of $a$ as a function of time.}
\label{fig:simpleplot2}
\end{figure}

\begin{figure}[bth]
\centering
\includegraphics[width=0.5\textwidth]{billion_figs/eccentricity.pdf}
\caption{This is the plot of $e$ as a function of time.}
\label{fig:simpleplot2}
\end{figure}

\begin{figure}[bth]
\centering
\includegraphics[width=0.5\textwidth]{billion_figs/energy.pdf}
\caption{This is the plot of $E$ as a function of time.}
\label{fig:simpleplot2}
\end{figure}

\begin{figure}[bth]
\centering
\includegraphics[width=0.5\textwidth]{billion_figs/angular_momentum.pdf}
\caption{This is the plot of $L_z$ as a function of time.}
\label{fig:simpleplot2}
\end{figure}

\subsection{Inspiral Analysis}

Having established the correctness of our simulation code by testing the Hulse-Taylor system, we next moved on to applying the simulation to the study of gravitational waves. The wave strains are most usefuly observed near the end of the system inspiral, so the next step was to apply the simulation to the final seconds prior to the system reaching the ``solution'' of $a(t)$. Fortunately, near the end of the inspiral, it is safe to assume that $e=0$, so the final semi-major axis may be found. Having done this, the simulation was then focused on the final few hundred seconds of the inspiral. \\

In Figure 5, one can see a comparison of the numerical result of our simulation of the final inspiral as compared to the analytic result (possible due to the assumption that $e=0$ in this range). This shows a high degree of accuracy of the simulation's results for the semi-major axis, so we may proceed, confident in out procedure, to analyzing the wave strain results. \\

Figures 6 and 7 show the wave strains plotted as a function of time, with Figure 7 showing a much narrower time domain. In this case, the actual waveforms of the strains can be seen. Figure 8 shows how the frequency of the gravitational waves change with time, and the frequency spectrum can be seen in Figure 9.

\begin{figure}[bth]
\centering
\includegraphics[width=0.5\textwidth]{inspiral_figs/inspiral_axs.pdf}
\caption{This is the plot of the semi-major axis as a function of time, leading up to the analytic solution for $a(t)$, assuming that the eccentricity is 0.}
\label{fig:simpleplot2}
\end{figure}

\begin{figure}[bth]
\centering
\includegraphics[width=0.5\textwidth]{inspiral_figs/inspiral_strains1.pdf}
\caption{This is the plot of inspiral gravitational wave strains as a function of time, leading up to the analytic solution.}
\label{fig:simpleplot2}
\end{figure}

\begin{figure}[bth]
\centering
\includegraphics[width=0.5\textwidth]{inspiral_figs/inspiral_strains2.pdf}
\caption{This is another plot of the inspiral gravitational wave strains as a function of time, except on a shorter timescale, such that the waveforms of the strains are visible.}
\label{fig:simpleplot2}
\end{figure}

\begin{figure}[bth]
\centering
\includegraphics[width=0.5\textwidth]{inspiral_figs/inspiral_freqs.pdf}
\caption{This is the plot of the gravitational wave frequency as a function of time in the time leading up to the conclusion of the inspiral.}
\label{fig:simpleplot2}
\end{figure}

\begin{figure}[bth]
\centering
\includegraphics[width=0.5\textwidth]{inspiral_figs/inspiral_strain_spectrum.pdf}
\caption{This is the plot of the frequency spectrum of the gravitational wave strain.}
\label{fig:simpleplot2}
\end{figure}

\section{Conclusion}

Using the results of Peters 1964, one can find expressions for the rates of change of the semi-major axis and eccentricity for a system of compact binaries with respect to time. By applying these results in a simulation using a Runge-Kutta 4 integration scheme, we are able to accurately recreated the predicted inspirals of such a binary system. \\

Having developed an accurate means for determining the orbits of a binary system, the simulation can then be applied to studying the gravitational wave emission of an inspiralling binary. By setting the eccentricity to 0 (valid near the end of the inspiral), the final regime of the inspiral can be studied in detail, producing accurate descriptions of the gravitational wave emissions.

\end{document}
